\documentclass[10pt,twocolumn]{article}
\usepackage[utf8]{inputenc}
\usepackage[margin=0.2in]{geometry}
\usepackage[compact]{titlesec}
\titlespacing{\section}{0pt}{2ex}{1ex}
\titlespacing{\subsection}{0pt}{1.5ex}{1ex}
\titlespacing{\subsubsection}{0pt}{0.5ex}{0ex}
\usepackage{amsmath,amsthm,amssymb,MnSymbol}
\setlength{\parindent}{0in}
\renewenvironment{proof}{{\bfseries Proof}}{\\}
\newcommand{\newlinetab}[0]{$\text{ }\hspace{3mm}$}

%--------------------------------------------------
% Intermediate Statistics 705 Cheatsheet for Test 3
%--------------------------------------------------

\begin{document}

\subsection*{Aymptotic (Large Sample) Theory}
A random sequence $A_{n}$ is:
\begin{flalign}
    &(a) \hspace{2mm} o_{p}(1) \text{ if } A_{n} \xrightarrow{p} 0\\
    &(b) \hspace{2mm} o_{p}(B_{n}) \text{ if } A_{n}/B_{n} \xrightarrow{p} 0\\
    &(c) \hspace{2mm} O_{p}(1) \text{ if } \forall \epsilon>0, \exists M :
        \lim_{n\rightarrow \infty} \mathbb{P}(|A_{n}|>M)<\epsilon \\
    &(d) \hspace{2mm} O_{p}(B_{n}) \text{ if } A_{n}/B_{n} = O_{p}(1)
\end{flalign}
If $Y_{n} \rightsquigarrow Y \implies Y_{n}=O_{p}(1)$\\
If $\sqrt{n}(Y_{n}-c)\rightsquigarrow Y$ $\implies$ $Y_{n}=O_{p}(1/\sqrt{n})$

\subsubsection*{Distances Between Distributions}
For distributions $P$ and $Q$ with pdfs $p$ and $q$:
\begin{flalign}
    &(a) \hspace{2mm} V(P,Q) = \sup_{A} |P(A)-Q(A)| \hspace{3mm} \text{\textbf{total variation} distance}\\
    &(b) \hspace{2mm} K(P,Q) = \int p \text{log}(p/q) \hspace{3mm} \text{\textbf{Kullback-Leibler} divergence}\\
    &(c) \hspace{2mm} d_{2}(P,Q) = \int (p-q)^{2} \hspace{3mm} \mathbf{L_{2}} \text{ distance}
\end{flalign}
A model is \textbf{identifiable} if: $\theta_{1} \neq \theta_{2}$ $\implies$ $K(\theta_{1},\theta_{2})>0$.

\subsubsection*{Consistency}
$\hat{\theta}_{n} = T(X^{n})$ is \textbf{consistent} for $\theta$ if $\hat{\theta}_{n} \xrightarrow{p} \theta$
    (ie if $\hat{\theta}_{n} - \theta = o_{p}(1)$).\\
To show consistency, can show: $\text{Bias}^{2}(\hat{\theta}_{n}) + \text{Var}(\hat{\theta}_{n}) \rightarrow 0$.\\
The MLE is consistent under regularity conditions.\\
MLE not consistent when number of params (or support?) grows.

\subsubsection*{Score and Fisher Information}
The \textbf{score function} is $S(\theta) = \frac{\partial}{\partial\theta} l(\theta) = \frac{\partial}{\partial\theta} \sum_{i=1}^{n} \text{log} \hspace{1pt} p(x_{i}|\theta)$.\\
The \textbf{Fisher information} is defined as
\begin{equation}
    I_{n}(\theta) = \mathbb{E}_{\theta} \left[ S(\theta)^{2} \right] = \text{Var}_{\theta} \left[ S(\theta) \right] 
        = -\mathbb{E}_{\theta} \left[ \frac{\partial^{2}}{\partial\theta^{2}} l(\theta) \right]
        %= -\mathbb{E}_{\theta} \left[ \frac{\partial}{\partial\theta} S(\theta) \right]
\end{equation}
    \newlinetab and $I_{n}(\theta) = -n\mathbb{E} \left[ \frac{\partial^{2}}{\partial\theta^{2}} \text{log}\hspace{1pt} p(X_{1};\theta) \right] = nI_{1}(\theta)$.\\
The \textbf{observed information} $\hat{I}_{n}(\theta) = -\sum_{i}\frac{\partial^{2}}{\partial\theta^{2}} \text{log}\hspace{1pt}p(X_{i};\theta)$.\\
    \newlinetab Vector case: $S(\theta) = \left[ \frac{\partial l(\theta)}{\partial \theta_{i}} \right]_{i=1,\ldots,K}$ \hspace{1mm}
        $I_{ij} = -\mathbb{E}_{\theta}\left[\frac{\partial^{2} l(\theta)}{\partial\theta_{i}\partial\theta_{j}}\right]_{i,j=1,\ldots,K}$

\subsubsection*{Efficiency and Robustness}
For an estimator $\hat{\theta}_{n}(X^{n})$ of $\theta$, where $X^{n} \stackrel{\text{iid}}{\sim} p(x|\theta)$:\\
If $\sqrt{n}(\hat{\theta}_{n} - \theta) \rightsquigarrow \mathcal{N}(0,v^{2})$, then $v^2$ is the \textbf{asymptotic-Var($\hat{\theta}_{n}$)}.\\
    \newlinetab E.g. for $\hat{\theta}_{n} = \overline{X}_{n}$:
        \hspace{1pt} $v^{2} = \sigma^{2} = \text{Var}(X_{i}) = \lim_{n \rightarrow \infty} n\text{Var}(\overline{X}_{n})$.\\
    \newlinetab In general, asymptotic-Var($\hat{\theta}_{n}$) $v^{2}$ $\neq$ $\lim_{n \rightarrow \infty} n\text{Var}(\hat{\theta}_{n})$.\\
    \newlinetab We will use approx: $\text{Var}(\hat{\theta}_{n}) \approx v^{2}/n$.\\
For param $\tau(\theta)$, $v(\theta) = \frac{|\tau'(\theta)|^{2}}{I_{1}(\theta)}$ is the \textbf{Cramer-Rao lower bound}.\\
    \newlinetab for most estimators $v^{2} \geq v(\theta)$.\\
If $\sqrt{n}(\hat{\theta}_{n}-\tau(\theta)) \rightsquigarrow \mathcal{N}(0,v(\theta))$ (ie if $v^{2} = v(\theta)$) $\implies \hat{\theta}_{n}$ \textbf{efficient}.\\
    \newlinetab usually, $\sqrt{n}(\tau(\hat{\theta}_{\text{mle}}) - \tau(\theta)) \rightsquigarrow \mathcal{N}(0,v(\theta))$ $\implies$ MLE efficient.\\
The \textbf{standard error} of \textbf{efficient} $\hat{\theta}_{n}$ is $se = \sqrt{\text{Var}(\hat{\theta}_{n})} \approx \sqrt{\frac{1}{I_{n}(\theta)}}$.\\
The \textbf{estimated standard error} of \textbf{efficient} $\hat{\theta}_{n}$ is $\hat{se} \approx \sqrt{\frac{1}{I_{n}(\hat{\theta}_{n})}}$.\\
    \newlinetab For efficient $\hat{\theta}_{n}$, $\hat{\tau} = \tau(\hat{\theta}_{n})$, $se \approx \sqrt{\frac{|\tau'(\theta)|^{2}}{I_{n}(\theta)}}$,
        and $\hat{se} \approx \sqrt{\frac{|\tau'(\hat{\theta}_{n})|^{2}}{I_{n}(\hat{\theta}_{n})}}$.\\
In general, \textbf{asymptotic normality} is when:\\
    \newlinetab $\frac{\hat{\theta}_{n} - \mathbb{E}(\hat{\theta}_{n})}{\sqrt{\text{Var}(\hat{\theta}_{n})}} \rightsquigarrow \mathcal{N}(0,1)$
        $\implies$ $\hat{\theta}_{n} \rightsquigarrow \mathcal{N}(\mathbb{E}(\hat{\theta}_{n}), \text{Var}(\hat{\theta}_{n}))$. \\
If $\sqrt{n}(W_{n}-\tau(\theta)) \rightsquigarrow \mathcal{N}(0,\sigma^{2}_{W})$ and
    $\sqrt{n}(V_{n}-\tau(\theta)) \rightsquigarrow \mathcal{N}(0,\sigma^{2}_{V})$ \\
    $\text{ }\hspace{2mm}$$\implies$ \textbf{asymptotic relative efficiency} $\text{ARE}(V_{n},W_{n}) = \sigma^{2}_{W} / \sigma^{2}_{V}$.\\
Often there is a tradeoff between efficiency and robustness. (?)

\newpage
\subsection*{Hypothesis Testing}
\textbf{Null hypothesis} $H_{0}: \theta \in \Theta_{0}$, \textbf{alternative} $H_{1}: \theta \in \Theta_{1}$.\\
\textbf{Type I error}: If $H_{0}$ true but we reject $H_{0}$.\\
To construct a test:
\begin{equation}
    \begin{split}
        &1. \text{ Choose a test statistic } W = W(X_{1},\ldots,X_{n})\\
        &2. \text{ Choose a rejection region } R\\
        &3. \text{ If } W\in R, \text{ reject } H_{0} \text{ otherwise retain } H_{0}
    \end{split}
\end{equation}
For rejection region $R$, the \textbf{power function} $\beta(\theta) = \mathbb{P}_{\theta}(X^{n} \in R)$.\\
Want \textbf{level-$\mathbf{\alpha}$} test ($\sup_{\theta \in \Theta_{0}} \beta(\theta) \leq \alpha$) that maximizes $\beta(\theta\in\Theta_{1})$.\\
A level-$\alpha$ test with power fn $\beta$ is \textbf{uniformly most powerful} if:
    \newlinetab$\beta(\theta) \geq \beta'(\theta)$ $\forall \theta \in \Theta_{1}$ $\forall \beta'\neq\beta$.
\subsubsection*{Neyman-Pearson Test}
For simple $H_{0}: \theta=\theta_{0}$ and $H_{1}: \theta=\theta_{1}$, reject $H_{0}$ if $\frac{L(\theta_{1})}{L(\theta_{0})} > k$.
    \newlinetab where $k$ chosen s.t. $\mathbb{P}(\frac{L(\theta_{1})}{L(\theta_{0})} > k) = \alpha$.
\subsubsection*{Wald Test}
For $H_{0}: \theta=\theta_{0}$ and $H_{1}: \theta\neq\theta_{0}$, reject $H_{0}$ if $\left| \frac{\hat{\theta}_{n} - \theta_{0}}{se} \right| > z_{\alpha/2}$.\\
    \newlinetab where $z_{\alpha/2}$ is the inverse standard-normal CDF of $1-\frac{\alpha}{2}$. \\%, ie $\Phi^{-1}(1-\alpha/2)$
    \newlinetab and $\hat{\theta}_{n}$ is an unbiased estimator for $\theta$.\\
    \newlinetab and $se = \sqrt{\text{Var}(\hat{\theta}_{n})}$. Can also use $\hat{se} =_{\text{eg.}} \sqrt{S_{n}^{2}/n}$.\\
    \newlinetab and if $\hat{\theta}_{n}$ efficient, can approx: $se \approx \sqrt{\frac{1}{I_{n}(\theta)}}$ or $\hat{se} \approx \sqrt{\frac{1}{I_{n}(\hat{\theta}_{n})}}$.

\subsubsection*{Likelihood Ratio Test}
For $H_{0}: \theta\in\Theta_{0}$ and $H_{1}: \theta\notin\Theta_{0}$, reject $H_{0}$ if $\lambda(x^{n}) = \frac{L(\hat{\theta}_{0})}{L(\hat{\theta})} \leq c$.
    \newlinetab where $L(\hat{\theta}_{0}) = \sup_{\theta\in\Theta_{0}}L(\theta)$ and $L(\hat{\theta}) = \sup_{\theta\in\Theta}L(\theta)$.\\
    \newlinetab and $c$ chosen s.t. $\mathbb{P}(\lambda(x^{n}) \leq c) = \alpha$.\\
    \newlinetab \textbf{Thm:} under $H_{0} : \theta=\theta_{0}$ $\implies$ $W_{n} = -2\text{log}\lambda(X^{n}) \rightsquigarrow \chi_{1}^{2}$\\
    \newlinetab\newlinetab $\implies$ reject $H_{0}$ if $W_{n}>\chi_{1,\alpha}^{2}$.\\ %\lim_{n\rightarrow\infty} \mathbb{P}_{\theta_{0}}(W_{n} > \chi_{1,\alpha}^{2}) = \alpha$.\\
        \newlinetab\newlinetab Also: for $\theta=(\theta_{1},\ldots,\theta_{k})$, if $H_{0}$ fixes some of the parameters\\
        \newlinetab\newlinetab $\implies$ $-2\text{log}\lambda(X^{n}) \rightsquigarrow \chi_{\nu}^{2}$, where $\nu = \text{dim}(\Theta) - \text{dim}(\Theta_{0})$.

\subsubsection*{P-Values}
The \textbf{p-value} $p(x^{n})$ is the smallest $\alpha$-level s.t. we reject $H_{0}$.\\
\textbf{Thm:} For a test of the form: reject $H_{0}$ when $W(x^{n})>c$,\\
    $\text{ }\hspace{1mm}$ $\implies$ $p(x^{n}) = \underset{\theta\in\Theta_{0}}{\sup} \mathbb{P}_{\theta}(W(X^{n}) \geq W(x^{n}))$
    $=$ $\underset{\theta\in\Theta_{0}}{\sup} [1 - F(W(x^{n})|\theta)]$.\\
\textbf{Thm:} Under $H_{0}:\theta=\theta_{0}$,\hspace{2mm}$p(x^{n}) \sim \text{Unif}(0,1)$.

\subsubsection*{Permutation Test}
$X^{n} \sim F$, $Y^{m} \sim G$, $H_{0}:F=G$, $H_{1}:F \neq G$\\
Let $Z=(X^{n},Y^{m})$ and $L=(1,\ldots,1,2,\ldots,2)$.\\
Let $W = g(L,Z) = |(\text{ave of 1 labeled pts}) - (\text{ave of 2 labeled pts})|$.
Let $p = \frac{1}{N!}\sum_{\pi} \mathbb{I}\left( g(L_{\pi},Z) > g(L,Z) \right)$ $\implies$ reject $H_{0}$ when $p<\alpha$.

\subsection*{Confidence Intervals}
We want a $1-\alpha$ \textbf{confidence interval} $C_{n} = [L(X^{n}),U(X^{n})]$ s.t.\\
    \newlinetab$\mathbb{P}_{\theta} \left( L(X^{n}) \leq \theta \leq U(X^{n}) \right)$ $\geq$ $1-\alpha$, \hspace{1mm} $\forall \theta\in\Theta$.\\
Generally, a $1-\alpha$ \textbf{confidence set} $C_{n}$ is a random set $C_{n} \subset \Theta$ s.t.\\
    \newlinetab $\inf_{\theta\in\Theta}\mathbb{P}_{\theta}\left( \theta \in C_{n}(X^{n}) \right) \geq 1-\alpha$.\\

\subsubsection*{Using Probability Inequalities}
Prob inequalities give (for eg.) $\mathbb{P}(|\hat{\theta}_{n} - \theta| > \epsilon) \leq g(\exp^{-f(\epsilon)}) \underset{\text{set to}}{=} \alpha$.
    $\text{ }\hspace{1pt}$ solving for $\epsilon$: $\mathbb{P} \left( |\hat{\theta}_{n} - \theta| > \tilde{f}(\alpha) \right) \leq \alpha \Rightarrow C_{n}$
        $=$ $\left( \hat{\theta}-\tilde{f}(\alpha), \hat{\theta}+\tilde{f}(\alpha) \right)$.\\

\subsubsection*{Inverting a Test}
In level-$\alpha$ tests $\mathbb{P}_{\theta_{0}}(T(x^{n}) \in R) = \alpha$ $\Rightarrow$ let $C_{n} = \{ \theta : T(x^{n}) \in A(\theta) \}$.
    $\text{ }\hspace{1mm}$where $A(\theta) = \{ T(x^{n}) \notin R \text{ s.t. } \theta = \theta_{0} \}$ (accept region if $\theta$ is null).\\
For Wald: $C_{n} = W_{n} \pm (z_{\alpha/2} \times se) = W_{n} \pm z_{\alpha/2}\frac{\sigma}{\sqrt{n}}$.\\
For LRT: $C_{n} = \{ \theta : \frac{L(\theta)}{L(\hat{\theta})} > c \}$ (for test where reject $H_{0}$ if $\frac{L(\theta_{0})}{L(\hat{\theta})} \leq c$).

\subsubsection*{Pivots}
$Q(X^{n},\theta)$ a \textbf{pivot} if the distribution of $Q$ does not depend on $\theta$.\\
Find $a$, $b$ s.t. $\mathbb{P}_{\theta}(a \leq Q(X^{n},\theta) \leq b) \geq 1-\alpha$, $\forall \theta$.\\
\newlinetab $\implies$ $C_{n} = \{ \theta : a \leq Q(X^{n},\theta) \leq b) \geq 1-\alpha \}$.


% --------------
% Older material
% --------------

%%%%%%%%%%%%%%%%%%%%%%%
% from test2_cheatsheet
%%%%%%%%%%%%%%%%%%%%%%%

\subsection*{Random Samples}
For $X_{1},\ldots,X_{n} \sim F$ a \textbf{statistic} is any $T = g(X_{1},\ldots,X_{n})$.\\
E.g. $\overline{X}_{n}$, $S_{n} = \sum_{i}(X_{i}-\overline{X}_{n})^{2} / (n-1)$, $\left(X_{(1)},\ldots,X_{(n)}\right)$\\
\textbf{Notes:} $\mathbb{E}(\overline{X}_{n}) = \mathbb{E}(X_{i})$, $\text{Var}(\overline{X}_{n}) = \text{Var}(X_{i})/n$, $\mathbb{E}(S_{n})^{2} = \text{Var}(X_{i})$, $X_{1,\ldots,n}$ $\sim$ $\text{Bern}(p)$ $\implies$ $\sum_{i} X_{i}$ $\sim$ $\text{Bin}(n,p)$, $X_{1,\ldots,n} \sim \text{Exp}(\beta)$ $\implies$ $\sum_{i}X_{i}$ $\sim$ $\Gamma(n,\beta)$, $X_{1,\ldots,n} \sim \mathcal{N}(0,1)$ $\implies$ $\sum_{i}X_{i}^{2} \sim \chi_{n}$. \\
\textbf{Thm. 1}: $X_{1},\ldots,X_{n} \sim \mathcal{N}(\mu,\sigma^{2})$ $\implies$  $\overline{X}_{n} \sim \mathcal{N}(\mu, \sigma^{2}/n)$.

\subsection*{Convergence}
$X,X_{1},X_{2},\ldots$ random variables.\\
(1) $X_{n}$ converges \textbf{almost surely} $X_{n} \xrightarrow{a.s.} X$ if $\forall \epsilon>0$
\begin{equation}
    \mathbb{P}(\lim_{n\rightarrow\infty} |X_{n}-X| < \epsilon) = 1
\end{equation}
(2) $X_{n}$ converges \textbf{in probability} $X_{n} \xrightarrow{p} X$ if $\forall \epsilon>0$
\begin{equation}
    \lim_{n\rightarrow\infty} \mathbb{P}(|X_{n}-X| < \epsilon) = 1 
\end{equation}
(3) $X_{n}$ converges \textbf{in quadratic mean} $X_{n} \xrightarrow{qm} X$ if
\begin{equation}
    \lim_{n\rightarrow\infty} \mathbb{E}[(X_{n}-X)^{2}] = 0
\end{equation}
(4) $X_{n}$ converges \textbf{in distribution} $X_{n} \rightsquigarrow X$ if
\begin{equation}
    \lim_{n\rightarrow\infty} F_{X_{n}}(t) = F_{X}(t)
\end{equation}
$\forall t$ on which $F_{X}$ is continuous.\\

\textbf{Thm 7:} Conv. a.s. and in q.m. imply conv. in prob. All three imply conv. in distribution. Conv. in distribution to a point-mass also implies conv. in prob.\\
%Ex from class: Showed conv. in prob $\not\implies$ conv. a.s.. Showed conv. in prob $\not\implies$ conv. in q.m.. Showed conv. in distro $\not\implies$ conv. in prob.

\textbf{Thm 10a:} $X$,$X_{n}$,$Y$,$Y_{n}$ random variables. Then
\begin{flalign}
    (a)& \hspace{2mm} X_{n} \xrightarrow{p} X, Y_{n} \xrightarrow{p} Y \implies X_{n} + Y_{n} \xrightarrow{p} X + Y \\
    (b)& \hspace{2mm} X_{n} \xrightarrow{p} X, Y_{n} \xrightarrow{p} Y \implies X_{n}Y_{n} \xrightarrow{p} XY \\
    (c)& \hspace{2mm} X_{n} \xrightarrow{qm} X, Y_{n} \xrightarrow{qm} Y \implies X_{n} + Y_{n} \xrightarrow{qm} X + Y
\end{flalign}

\textbf{Thm 10b (Slutzky's Thm):} $X$,$X_{n}$,$Y_{n}$ random variables. Then
\begin{flalign}
    (a)& \hspace{2mm} X_{n} \rightsquigarrow X, Y_{n} \rightsquigarrow c \implies X_{n} + Y_{n} \rightsquigarrow X + c \\
    (b)& \hspace{2mm} X_{n} \rightsquigarrow X, Y_{n} \rightsquigarrow c \implies X_{n}Y_{n} \rightsquigarrow cX
\end{flalign}

\textbf{Thm 12 (Law of Large Numbers):} $X_{1},\ldots,X_{n}$ iid, $\mathbb{E}(X_{i})=\mu$ $\implies$ $\overline{X}_{n} \xrightarrow{\text{qm}} \mu$.

\textbf{Thm 14 (CLT):} $X_{1},\ldots,X_{n}$ iid, $\mathbb{E}(X_{i})=\mu$ $\text{Var}(X_{i}) = \sigma^{2}$\\
$\implies$ $\sqrt{n}(\overline{X}_{n}-\mu)/\sigma \rightsquigarrow \mathcal{N}(0,1)$\\
$\implies$ $\overline{X}_{n} \rightsquigarrow \mathcal{N}(\mu,\sigma^{2}/n)$\\
$\implies$ $\sqrt{n}(\overline{X}_{n}-\mu)/S_{n}\rightsquigarrow \mathcal{N}(0,1)$

\textbf{Thm 18 (delta method):} If $\sqrt{n}(Y_{n}-\mu)/\sigma \rightsquigarrow \mathcal{N}(0,1)$, $g'(\mu) \neq 0$
$\implies$ $\sqrt{n}(g(Y_{n})-g(\mu))/|g'(\mu)|\sigma \rightsquigarrow \mathcal{N}(0,1)$\\
ie $Y_{n} \approx \mathcal{N}(\mu,\sigma^{2}/n)$ $\implies$ $g(Y_{n}) \approx \mathcal{N}(g(\mu),g'(\mu)^{2}\sigma^{2}/n)$


%%%%%%%%%%%%%%%%%%%%%%%
% from test1_cheatsheet
%%%%%%%%%%%%%%%%%%%%%%%

\newpage

\subsection*{Distributions}
Discrete distributions:
\begin{flalign}
(a)& \hspace{2mm} \text{Bernoulli} \hspace{2mm} f(x|p) = p^{x}(1-p)^{1-x}, \hspace{3mm} x \in \{0, 1\} \\
(b)& \hspace{2mm} \text{Binomial} \hspace{2mm} f(x|n,p) = {\binom{n}{x}} p^{x}(1-p)^{n-x}, \hspace{2mm} x \in \{0,1,\ldots,n\} \\
(c)& \hspace{2mm} \text{Poisson} \hspace{2mm} f(x|\lambda) = \frac{e^{-\lambda}\lambda^{x}}{x!}, \hspace{3mm} x \in \{0, 1, 2, \ldots\}
\end{flalign}

Continuous distributions:
\begin{flalign}
(a)& \hspace{2mm} \text{Uniform} \hspace{2mm} f(x|a,b) = \frac{1}{b-a}, \hspace{3mm} x \in [a,b] \\
(b)& \hspace{2mm} \text{Normal} \hspace{2mm} f(x|\mu,\sigma^{2}) = \frac{1}{\sigma\sqrt{2\pi}}e^{-(x-\mu)^{2}/(2\sigma^{2})}, \hspace{2mm} x \in \mathbb{R} \\
(c)& \hspace{2mm} \text{Gamma} \hspace{2mm} f(x|\alpha,\beta) = \frac{1}{\Gamma(\alpha)\beta^{\alpha}}x^{\alpha-1}e^{-x/\beta}, \hspace{2mm} x \in \mathbb{R}_{+}, \alpha \hspace{1mm}\beta>0
\end{flalign}


\subsection*{Expected Values}
The \textbf{mean} or \textbf{expected value} of $g(X)$ is
\begin{equation}
\mathbb{E}(g(X)) = \int g(x)dF(x) = \int g(x)dP(x)
\end{equation}
Related properties and definitions:
\begin{flalign}
(a)& \hspace{2mm} \mu = \mathbb{E}(X) \\
(b)& \hspace{2mm} \mathbb{E}(\sum_{i} c_{i}g_{i}(X_{i})) = \sum_{i} c_{i} \mathbb{E}(g_{i}(X_{i})) \\
(c)& \hspace{2mm} \mathbb{E}\left(\prod_{i} X_{i} \right) = \prod_{i} \mathbb{E}(X_{i}), \hspace{4mm} X_{1}, \ldots, X_{n} \text{ indep't} \\
(d)& \hspace{2mm} Var(X) = \sigma^{2} = \mathbb{E}((X-\mu)^{2}) \hspace{4mm} \text{is the \textbf{variance} of X} \\
(e)& \hspace{2mm} Var(X) = \mathbb{E}(X^{2}) - \mu^{2} \\
(f)& \hspace{2mm} Var \left (\sum_{i} a_{i}X_{i} \right ) = \sum_{i} a_{i}^{2} Var(X_{i}), \hspace{4mm} X_{1}, \ldots, X_{n} \text{ indep't} \\
(g)& \hspace{2mm} Cov(X,Y) = \mathbb{E}((X-\mu_{X})(Y-\mu_{Y})) \hspace{2mm} \text{is the \textbf{covariance}} \\
(h)& \hspace{2mm} Cov(X,Y) = \mathbb{E}(XY) - \mu_{x}\mu_{Y} \\
(i)& \hspace{2mm} \rho(X,Y) = Cov(X,Y) / \sigma_{x}\sigma_{y}, \hspace{4mm} -1 \leq \rho(X,Y) \leq 1
\end{flalign}
The \textbf{conditional expectation} of Y given X is the random variable $g(X) = \mathbb{E}(Y|X)$, where
\begin{gather}
\mathbb{E}(Y|X=x) = \int y f(y|x)dy\\
\text{and} \hspace{2mm} f(y|x) = f_{X,Y}(x,y) / f_{X}(x)
\end{gather}
The \emph{Law of Total/Iterated Expectation} is
\begin{equation}
\mathbb{E}(Y) = \mathbb{E}[\mathbb{E}(Y|X)]
\end{equation}
The \emph{Law of Total Variance} is
\begin{equation}
Var(Y) = Var[\mathbb{E}(Y|X)] + \mathbb{E}[Var(Y|X)]
\end{equation}
The \emph{Law of Total Covariance} is
\begin{equation}
Cov(X,Y) = \mathbb{E}(Cov(X,Y|Z)) + Cov(\mathbb{E}(X|Z), \mathbb{E}(Y|Z))
\end{equation}


\end{document}
