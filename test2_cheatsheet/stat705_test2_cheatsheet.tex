\documentclass[10pt,twocolumn]{article}
\usepackage[utf8]{inputenc}
\usepackage[margin=0.2in]{geometry}
\usepackage[compact]{titlesec}
\titlespacing{\section}{0pt}{2ex}{1ex}
\titlespacing{\subsection}{0pt}{1.5ex}{1ex}
\titlespacing{\subsubsection}{0pt}{0.5ex}{0ex}
\usepackage{amsmath,amsthm,amssymb,MnSymbol}
\setlength{\parindent}{0in}
\renewenvironment{proof}{{\bfseries Proof}}{\\}

%--------------------------------------------------
% Intermediate Statistics 705 Cheatsheet for Test 2 
%--------------------------------------------------

\begin{document}


\subsection*{Probability Inequalities}
\textbf{Thm 1 (Gaussian Tail Inequality):}
Let $X \sim \mathcal{N}(0,1)$. Then
\begin{equation}
    \mathbb{P}(|X| > \epsilon) \leq \frac{2}{\epsilon}e^{-\epsilon^{2}/2}
\end{equation}
Additionally:
\begin{equation}
\mathbb{P}(|\overline{X}_{n}| > \epsilon) \leq \frac{1}{\sqrt{n}\epsilon}e^{-n\epsilon^{2}/2}
\end{equation}
    
\textbf{Thm 2 (Markov Inequality):}
Let X be a non-negative random variable s.t. $\mathbb{E}(X)$ exists. Then $\forall$ $t>0$
\begin{equation}
    \mathbb{P}(X>t) \leq \frac{\mathbb{E}(X)}{t}
\end{equation}

\textbf{Thm 3 (Chebyshev's Inequality):}
Let $\mu = \mathbb{E}(X)$ and $\sigma^{2} = \text{Var}(X)$. Then
\begin{gather}
    \mathbb{P}(|X-\mu| \geq t) \leq \frac{\sigma^{2}}{t^{2}} \\
    \mathbb{P}(|(X-\mu)/\sigma| \geq t) \leq \frac{1}{t^{2}}
\end{gather}

\textbf{Lemma 4:}
Let $\mathbb{E}(X) = 0$ and $a \leq X \leq b$. Then
\begin{equation}
    \mathbb{E}(e^{tX}) \leq e^{t^{2}(b-a)^{2}/8}
\end{equation}

\textbf{Lemma 5:}
Let $X$ be any random variable. Then
\begin{equation}
    \mathbb{P}(X>\epsilon) \leq \inf_{t \geq 0} e^{-t\epsilon} \mathbb{E}(e^{tX})
    %\mathbb{P}(X>\epsilon) \leq \underset{t \geq 0}{\text{inf}} e^{-t\epsilon} \mathbb{E}(e^{tX})
\end{equation}

\textbf{Thm 6 (Hoeffding's Inequality):}
$X_{1},\ldots,X_{n}$ iid, $\mathbb{E}(X_{i}) = \mu$, $a \leq X_{i} \leq b$. Then $\forall \epsilon >0$
\begin{equation}
    \mathbb{P}(|\overline{X} - \mu| \geq \epsilon) \leq 2e^{-2n\epsilon^{2}/(b-a)^{2}}
\end{equation}

\textbf{Thm 9 (McDiarmid):} $X_{1},\ldots,X_{n}$ indep't. If\\
$\sup_{x_{1},\ldots,x_{n},x'_{i}} \left| g(x_{1},\ldots,x_{n}) - g_{i}^{*}(x_{1},\ldots,x_{n}) \right|$ $\leq c_{i}$ $\forall i$, $\implies$
\begin{equation}
    \mathbb{P} \left(g(X_{1},\ldots,X_{n})-\mathbb{E}(g(X_{1},\ldots,X_{n})) \geq \epsilon \right) \leq e^{-2\epsilon^{2}/\sum_{i}c_{i}^{2}}
\end{equation}
where $g_{i}^{*} = g$ with $x_{i}$ replaced by $x'_{i}$.

\textbf{Thm 12 (Cauchy-Schwartz inequality):}

\textbf{Thm 13 (Jensen's inequality):}

\textbf{Ex 15 (Kullback Leibler distance):}

\textbf{Thm 18:}

\textbf{$O_{p}$ and $o_{p}$:} $X_{n} = o_{p}(1)$ if $\forall$ $\epsilon>0$, $\lim_{n\rightarrow\infty} \mathbb{P}(|X_{n}|>\epsilon) = 0$.\\
$X_{n} = O_{p}(1)$ if $\forall$ $\epsilon>0$, $\exists$ $C>0$ s.t. $\lim_{n\rightarrow\infty} \mathbb{P}(|X_{n}|>C) \leq \epsilon$. \\
$X_{n} = o_{p}(a_{n})$ if $X_{n}/a_{n} = o_{p}(1)$ and $X_{n} = O_{p}(a_{n})$ if $X_{n}/a_{n} = O_{p}(1).$

\subsection*{Shattering}
Note: remember uniform bounds and union bound.\\
$F$ a finite set, $|F| = n$, and $G \subset F$. $\mathcal{A}$ is a class of sets.\\
$\mathcal{A}$ \textbf{picks out} $G$ if $\exists A \in \mathcal{A}$ s.t. $A \cap F = G$.\\
Let $S(\mathcal{A},F)$ $=$ $|\{G \subset F \text{ picked out by } \mathcal{A}\}|$ $\leq 2^{n}$.\\
$F$ is \textbf{shattered} by $\mathcal{A}$ if $S(\mathcal{A},F) = 2^{n}$ (ie if $\mathcal{A}$ picks out all $G \subset F$).\\
Let $\mathcal{F}_{n}$ be all finite sets with $n$ elements.\\
The \textbf{shatter coefficient} $s_{n}(\mathcal{A}) = \sup_{F \in \mathcal{F}_{n}} s(\mathcal{A},F) \leq 2^{n}$.\\
The \textbf{VC dimension} $d(\mathcal{A}) =$ the largest $n$ s.t. $s_{n}(\mathcal{A}) = 2^{n}$.\\
\textbf{Thm 5:} $\forall \epsilon>0$, $\mathbb{P}(\sup_{A \in \mathcal{A}} |P_{n}(A) - P(A)| > \epsilon) \leq 8 s_{n}(\mathcal{A})e^{-n\epsilon^{2}/32}$


\subsection*{Random Samples}
For $X_{1},\ldots,X_{n} \sim F$ a \textbf{statistic} is any $T = g(X_{1},\ldots,X_{n})$.\\
E.g. $\overline{X}_{n}$, $S_{n} = \sum_{i}(X_{i}-\overline{X}_{n})^{2} / (n-1)$, $\left(X_{(1)},\ldots,X_{(n)}\right)$\\
\textbf{Notes:} $\mathbb{E}(\overline{X}_{n}) = \mathbb{E}(X_{i})$, $\text{Var}(\overline{X}_{n}) = \text{Var}(X_{i})/n$, $\mathbb{E}(S_{n})^{2} = \text{Var}(X_{i})$, $X_{1,\ldots,n}$ $\sim$ $\text{Bern}(p)$ $\implies$ $\sum_{i} X_{i}$ $\sim$ $\text{Bin}(n,p)$, $X_{1,\ldots,n} \sim \text{Exp}(\beta)$ $\implies$ $\sum_{i}X_{i}$ $\sim$ $\Gamma(n,\beta)$, $X_{1,\ldots,n} \sim \mathcal{N}(0,1)$ $\implies$ $\sum_{i}X_{i}^{2} \sim \chi_{n}$. \\
\textbf{Thm. 1}: $X_{1},\ldots,X_{n} \sim \mathcal{N}(\mu,\sigma^{2})$ $\implies$  $\overline{X}_{n} \sim \mathcal{N}(\mu, \sigma^{2}/n)$.


\subsection*{Convergence}
$X,X_{1},X_{2},\ldots$ random variables.\\
(1) $X_{n}$ converges \textbf{almost surely} $X_{n} \xrightarrow{a.s.} X$ if $\forall \epsilon>0$
\begin{equation}
    \mathbb{P}(\lim_{n\rightarrow\infty} |X_{n}-X| < \epsilon) = 1
\end{equation}
(2) $X_{n}$ converges \textbf{in probability} $X_{n} \xrightarrow{p} X$ if $\forall \epsilon>0$
\begin{equation}
    \lim_{n\rightarrow\infty} \mathbb{P}(|X_{n}-X| < \epsilon) = 1 
\end{equation}
(3) $X_{n}$ converges \textbf{in quadratic mean} $X_{n} \xrightarrow{qm} X$ if
\begin{equation}
    \lim_{n\rightarrow\infty} \mathbb{E}[(X_{n}-X)^{2}] = 0
\end{equation}
(4) $X_{n}$ converges \textbf{in distribution} $X_{n} \rightsquigarrow X$ if
\begin{equation}
    \lim_{n\rightarrow\infty} F_{X_{n}}(t) = F_{X}(t)
\end{equation}
$\forall t$ on which $F_{X}$ is continuous.\\

\textbf{Thm 7:} Conv. a.s. and in q.m. imply conv. in prob. All three imply conv. in distribution. Conv. in distribution to a point-mass also implies conv. in prob.\\
%Ex from class: Showed conv. in prob $\not\implies$ conv. a.s.. Showed conv. in prob $\not\implies$ conv. in q.m.. Showed conv. in distro $\not\implies$ conv. in prob.

\textbf{Thm 10a:} $X$,$X_{n}$,$Y$,$Y_{n}$ random variables. Then
\begin{flalign}
    (a)& \hspace{2mm} X_{n} \xrightarrow{p} X, Y_{n} \xrightarrow{p} Y \implies X_{n} + Y_{n} \xrightarrow{p} X + Y \\
    (b)& \hspace{2mm} X_{n} \xrightarrow{p} X, Y_{n} \xrightarrow{p} Y \implies X_{n}Y_{n} \xrightarrow{p} XY \\
    (c)& \hspace{2mm} X_{n} \xrightarrow{qm} X, Y_{n} \xrightarrow{qm} Y \implies X_{n} + Y_{n} \xrightarrow{qm} X + Y
\end{flalign}

\textbf{Thm 10b (Slutzky's Thm):} $X$,$X_{n}$,$Y_{n}$ random variables. Then
\begin{flalign}
    (a)& \hspace{2mm} X_{n} \rightsquigarrow X, Y_{n} \rightsquigarrow c \implies X_{n} + Y_{n} \rightsquigarrow X + c \\
    (b)& \hspace{2mm} X_{n} \rightsquigarrow X, Y_{n} \rightsquigarrow c \implies X_{n}Y_{n} \rightsquigarrow cX
\end{flalign}

\textbf{Thm 12 (Law of Large Numbers):} $X_{1},\ldots,X_{n}$ iid, $\mathbb{E}(X_{i})=\mu$ $\implies$ $\overline{X}_{n} \xrightarrow{\text{qm}} \mu$.

\textbf{Thm 14 (CLT):} $X_{1},\ldots,X_{n}$ iid, $\mathbb{E}(X_{i})=\mu$ $\text{Var}(X_{i}) = \sigma^{2}$\\
$\implies$ $\sqrt{n}(\overline{X}_{n}-\mu)/\sigma \rightsquigarrow \mathcal{N}(0,1)$\\
$\implies$ $\overline{X}_{n} \rightsquigarrow \mathcal{N}(\mu,\sigma^{2}/n)$\\
$\implies$ $\sqrt{n}(\overline{X}_{n}-\mu)/S_{n}\rightsquigarrow \mathcal{N}(0,1)$

\textbf{Thm 18 (delta method):} If $\sqrt{n}(Y_{n}-\mu)/\sigma \rightsquigarrow \mathcal{N}(0,1)$, $g'(\mu) \neq 0$
$\implies$ $\sqrt{n}(g(Y_{n})-g(\mu))/|g'(\mu)|\sigma \rightsquigarrow \mathcal{N}(0,1)$\\
ie $Y_{n} \approx \mathcal{N}(\mu,\sigma^{2}/n)$ $\implies$ $g(Y_{n}) \approx \mathcal{N}(g(\mu),g'(\mu)^{2}\sigma^{2}/n)$

\textbf{Thm 18b (2nd order delta method):??} Should I include this?


\subsection*{Sufficiency}
If $X_{1},\ldots,X_{n} \sim p(x;\theta)$, $T$ \textbf{sufficient} for $\theta$ if $p(x^{n}|t;\theta)$ $=$ $p(x^{n}|t)$.
\textbf{Thm 9 (factorization):} for $X^{n} \sim p(x;\theta)$, $T(X^{n})$ sufficient for $\theta$ if the joint probability can be factorized as.
\begin{equation}
    p(x^{n};\theta) = h(x^{n}) \times g(t;\theta)
\end{equation}
$T$ is a \textbf{minimal sufficient statistic (MSS)} if $T$ is sufficient and $T = g(U)$ for all other sufficient stats $U$.\\
\textbf{Thm 15:} $T$ is a MSS if:
\begin{equation}
    \frac{p(y^{n};\theta)}{p(x^{n};\theta)} \text{ constant in $\theta$ } \iff T(y^{n}) = T(x^{n})
\end{equation}


\subsection*{Parametric Point Estimation}
make sure i've defined: $\mathbb{E}_{\theta}(\hat{\theta})$, bias, sampling distro, standard error, $\hat{\theta}_{n}$ consistent.\\
\textbf{Method of Moments:} Define equations\\
(a) \hspace{3mm} $(\sum_{i} X_{i})/n = \mathbb{E}_{\hat{\theta}}(X_{i})$\\
(b) \hspace{3mm} $(\sum_{i} X_{i}^{2})/n = \mathbb{E}_{\hat{\theta}}(X_{i}^{2})$\\
(c) \hspace{3mm} $\ldots$ \\
And solve for $\hat{\theta}$.\\
\textbf{Maximum Likelihood (MLE):} The MLE is
\begin{equation}
\hat{\theta} = \text{arg}\max_{\theta} L(\theta) = \text{arg}\max_{\theta} l(\theta)
\end{equation}
Often suffices to solve for $\theta$ in $\frac{\partial l(\theta)}{\partial \theta} = 0$.
The MLE is \textbf{equivariant} $\implies$ if $\eta = g(\theta)$ then $\hat{\eta} = g(\hat{\theta})$. \\
\textbf{Bayes Estimation:} For prior $\pi(\theta)$, choose 
\begin{equation}
    \hat{\theta} = \mathbb{E}(\theta|x^{n}) = \int \theta \pi(\theta|x^{n}) d\pi
\end{equation}
\textbf{Mean Squared Error (MSE):} The MSE is
\begin{equation}
    \text{MSE} = \mathbb{E}(\hat{\theta} - \theta)^{2} = \int (\hat{\theta}-\theta)^{2} p(x^{n};\theta)dx^{n} = \text{bias}({\hat{\theta}})^{2} + Var(\hat{\theta})
\end{equation}
Defs: $\text{\textbf{bias}}(\hat{\theta})$ $=$ $\mathbb{E}(\hat{\theta}) - \theta$. We say $\hat{\theta}$ is \textbf{consistent} if $\hat{\theta} = \hat{\theta}_{n} \xrightarrow{p} \theta$. The \textbf{standard error} of $\hat{\theta}$, $\text{se}(\hat{\theta})$, is the standard deviation of $\hat{\theta}$.\\


\subsection*{Risks and Estimators}
$L(\theta,\hat{\theta})$ is the \textbf{loss} of an estimator $\hat{\theta} = \hat{\theta}(x^{n})$ for $x^{n} \sim p(x^{n};\theta)$.\\
The \textbf{risk} of this $\hat{\theta}$ is
\begin{equation}
    R(\theta,\hat{\theta}) = \mathbb{E}[L(\theta,\hat{\theta})] = \int L(\theta,\hat{\theta}) p(x^{n};\theta) dx^{n}
\end{equation}
When $L(\theta,\hat{\theta}) = (\theta-\hat{\theta})^{2}$, the risk is the MSE.\\
The \textbf{max risk} of $\hat{\theta}$ over a set $\theta \in \Theta$ is
\begin{equation}
    \overline{R}(\hat{\theta}) = \sup_{\theta \in \Theta} R(\theta,\hat{\theta})
\end{equation}
%The \textbf{minimax risk} is
%\begin{equation}
    %R_{n} = \inf_{\hat{\theta}} \sup_{\theta \in \Theta} R(\theta,\hat{\theta})
%\end{equation}
The \textbf{minimax estimator} is
\begin{equation}
    \hat{\theta} = \text{arg}\inf_{\hat{\theta}} \overline{R}(\hat{\theta})
\end{equation}
The \textbf{Bayes risk} of $\hat{\theta}$ given a prior $\pi(\theta)$ is
\begin{equation}
    B_{\pi}(\hat{\theta}) = \int R(\theta,\hat{\theta}) \pi(\theta) d\theta
\end{equation}
The \textbf{posterior risk} of $\hat{\theta}$ given a prior $\pi(\theta)$ is
\begin{equation}
    r(\hat{\theta}|x^{n}) = \int L(\theta,\hat{\theta}) \pi(\theta|x^{n}) d\theta    
\end{equation}
where $\pi(\theta|x^{n}) = \frac{\mathbb{P}(x^{n};\theta)\pi(\theta)}{m(x^{n})}$ is the posterior over $\theta$.\\
The \textbf{Bayes estimator} is
\begin{equation}
    \hat{\theta} = \text{arg}\inf_{\hat{\theta}} B_{\pi}(\hat{\theta}) = \text{arg}\inf_{\hat{\theta}} r(\hat{\theta}|x^{n})
\end{equation}
which equals the posterior mean $\mathbb{E}(\theta|x^{n})$ when $L(\theta,\hat{\theta}) = (\theta-\hat{\theta})^{2}$, the posterior median when $L(\theta,\hat{\theta}) = |\theta-\hat{\theta}|$, and the posterior mode when $L(\theta,\hat{\theta}) = \mathbb{I}[\theta \neq \hat{\theta}]$.\\
\textbf{Thm 10:} If $\hat{\theta}$ is a Bayes estimator for some prior $\pi$ and $R(\theta,\hat{\theta})$ is constant, then $\hat{\theta}$ is a minimax estimator.\\
\textbf{Note:} The MLE is approximately minimax (as n increases, if dimension of the parameter is fixed).


\subsection*{Distributions}
Discrete distributions:
\begin{flalign}
(a)& \hspace{2mm} \text{Bernoulli} \hspace{2mm} f(x|p) = p^{x}(1-p)^{1-x}, \hspace{3mm} x \in \{0, 1\} \\
(b)& \hspace{2mm} \text{Binomial} \hspace{2mm} f(x|n,p) = {\binom{n}{x}} p^{x}(1-p)^{n-x}, \hspace{2mm} x \in \{0,1,\ldots,n\} \\
(c)& \hspace{2mm} \text{Poisson} \hspace{2mm} f(x|\lambda) = \frac{e^{-\lambda}\lambda^{x}}{x!}, \hspace{3mm} x \in \{0, 1, 2, \ldots\}
\end{flalign}

Continuous distributions:
\begin{flalign}
(a)& \hspace{2mm} \text{Uniform} \hspace{2mm} f(x|a,b) = \frac{1}{b-a}, \hspace{3mm} x \in [a,b] \\
(b)& \hspace{2mm} \text{Normal} \hspace{2mm} f(x|\mu,\sigma^{2}) = \frac{1}{\sigma\sqrt{2\pi}}e^{-(x-\mu)^{2}/(2\sigma^{2})}, \hspace{2mm} x \in \mathbb{R} \\
(c)& \hspace{2mm} \text{Gamma} \hspace{2mm} f(x|\alpha,\beta) = \frac{1}{\Gamma(\alpha)\beta^{\alpha}}x^{\alpha-1}e^{-x/\beta}, \hspace{2mm} x \in \mathbb{R}_{+}, \alpha \hspace{1mm}\beta>0
\end{flalign}


\subsection*{Expected Values}
The \textbf{mean} or \textbf{expected value} of $g(X)$ is
\begin{equation}
\mathbb{E}(g(X)) = \int g(x)dF(x) = \int g(x)dP(x)
\end{equation}
Related properties and definitions:
\begin{flalign}
(a)& \hspace{2mm} \mu = \mathbb{E}(X) \\
(b)& \hspace{2mm} \mathbb{E}(\sum_{i} c_{i}g_{i}(X_{i})) = \sum_{i} c_{i} \mathbb{E}(g_{i}(X_{i})) \\
(c)& \hspace{2mm} \mathbb{E}\left(\prod_{i} X_{i} \right) = \prod_{i} \mathbb{E}(X_{i}), \hspace{4mm} X_{1}, \ldots, X_{n} \text{ indep't} \\
(d)& \hspace{2mm} Var(X) = \sigma^{2} = \mathbb{E}((X-\mu)^{2}) \hspace{4mm} \text{is the \textbf{variance} of X} \\
(e)& \hspace{2mm} Var(X) = \mathbb{E}(X^{2}) - \mu^{2} \\
(f)& \hspace{2mm} Var \left (\sum_{i} a_{i}X_{i} \right ) = \sum_{i} a_{i}^{2} Var(X_{i}), \hspace{4mm} X_{1}, \ldots, X_{n} \text{ indep't} \\
(g)& \hspace{2mm} Cov(X,Y) = \mathbb{E}((X-\mu_{X})(Y-\mu_{Y})) \hspace{2mm} \text{is the \textbf{covariance}} \\
(h)& \hspace{2mm} Cov(X,Y) = \mathbb{E}(XY) - \mu_{x}\mu_{Y} \\
(i)& \hspace{2mm} \rho(X,Y) = Cov(X,Y) / \sigma_{x}\sigma_{y}, \hspace{4mm} -1 \leq \rho(X,Y) \leq 1
\end{flalign}
The \textbf{conditional expectation} of Y given X is the random variable $g(X) = \mathbb{E}(Y|X)$, where
\begin{gather}
\mathbb{E}(Y|X=x) = \int y f(y|x)dy\\
\text{and} \hspace{2mm} f(y|x) = f_{X,Y}(x,y) / f_{X}(x)
\end{gather}
The \emph{Law of Total/Iterated Expectation} is
\begin{equation}
\mathbb{E}(Y) = \mathbb{E}[\mathbb{E}(Y|X)]
\end{equation}
The \emph{Law of Total Variance} is
\begin{equation}
Var(Y) = Var[\mathbb{E}(Y|X)] + \mathbb{E}[Var(Y|X)]
\end{equation}
The \emph{Law of Total Covariance} is
\begin{equation}
Cov(X,Y) = \mathbb{E}(Cov(X,Y|Z)) + Cov(\mathbb{E}(X|Z), \mathbb{E}(Y|Z))
\end{equation}


\end{document}
