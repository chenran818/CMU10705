\documentclass[10 pt,twocolumn]{article}
\usepackage[utf8]{inputenc}
\usepackage[margin=0.2in]{geometry}
\usepackage[compact]{titlesec}
\titlespacing{\section}{0pt}{2ex}{1ex}
\titlespacing{\subsection}{0pt}{1.5ex}{1ex}
\titlespacing{\subsubsection}{0pt}{0.5ex}{0ex}
\usepackage{amsmath,amsthm,amssymb,MnSymbol}
\setlength{\parindent}{0in}
\renewenvironment{proof}{{\bfseries Proof}}{\\}
\newcommand{\newlinetab}[0]{$\text{ }\hspace{3mm}$}

%--------------------------------------------------
% Intermediate Statistics 705 Cheatsheet for Test 2 
%--------------------------------------------------

\begin{document}


\subsection*{Probability Inequalities}
\textbf{Thm 1 (Gaussian Tail Inequality):}\\
Let $X \sim \mathcal{N}(0,1)$. Then\\
Additionally:

\textbf{Thm 2 (Markov Inequality):}
Let X be a non-negative random variable s.t. $\mathbb{E}(X)$ exists. 
 \\Then $\forall$ $t>0$

\textbf{Thm 3 (Chebyshev's Inequality):}
Let $\mu = \mathbb{E}(X)$ and $\sigma^{2} = \text{Var}(X)$. 
\\ Then:

\textbf{Lemma 4:}
Let $\mathbb{E}(X) = 0$ and $a \leq X \leq b$. 
 \\Then

\textbf{Lemma 5:}
Let $X$ be any random variable. 
 \\Then

\textbf{Thm 6 (Hoeffding's Inequality):}
$X_{1},\ldots,X_{n}$ iid, $\mathbb{E}(X_{i}) = \mu$, $a \leq X_{i} \leq b$. 
\\Then $\forall \epsilon >0$

\textbf{Thm 9 (McDiarmid):} $X_{1},\ldots,X_{n}$ indep't. If\\
$\sup_{x_{1},\ldots,x_{n},x'_{i}} \left| g(x_{1},\ldots,x_{n}) - g_{i}^{*}(x_{1},\ldots,x_{n}) \right|$ $\leq c_{i}$ $\forall i$, $\implies$
\begin{equation}
    \mathbb{P} \left(g(X_{1},\ldots,X_{n})-\mathbb{E}(g(X_{1},\ldots,X_{n})) \geq \epsilon \right) \leq e^{-2\epsilon^{2}/\sum_{i}c_{i}^{2}}
\end{equation}
where $g_{i}^{*} = g$ with $x_{i}$ replaced by $x'_{i}$.

\textbf{Thm 12 (Cauchy-Schwartz inequality):}

\textbf{Thm 13 (Jensen's inequality):}

\textbf{Ex 15 (Kullback Leibler distance):}

\textbf{Thm 18:}

\textbf{$O_{p}$ and $o_{p}$:} $X_{n} = o_{p}(1)$ if $\forall$ $\epsilon>0$, $\lim_{n\rightarrow\infty} \mathbb{P}(|X_{n}|>\epsilon) = 0$.\\
$X_{n} = O_{p}(1)$ if $\forall$ $\epsilon>0$, $\exists$ $C>0$ s.t. $\lim_{n\rightarrow\infty} \mathbb{P}(|X_{n}|>C) \leq \epsilon$. \\
$X_{n} = o_{p}(a_{n})$ if $X_{n}/a_{n} = o_{p}(1)$ and $X_{n} = O_{p}(a_{n})$ if $X_{n}/a_{n} = O_{p}(1).$

\subsection*{Shattering}
Note: remember uniform bounds and union bound.\\
$F$ a finite set, $|F| = n$, and $G \subset F$. $\mathcal{A}$ is a class of sets.\\
$\mathcal{A}$ \textbf{picks out} $G$ if $\exists A \in \mathcal{A}$ s.t. $A \cap F = G$.\\
Let $S(\mathcal{A},F)$ $=$ $|\{G \subset F \text{ picked out by } \mathcal{A}\}|$ $\leq 2^{n}$.\\
$F$ is \textbf{shattered} by $\mathcal{A}$ if $S(\mathcal{A},F) = 2^{n}$ (ie if $\mathcal{A}$ picks out all $G \subset F$).\\
Let $\mathcal{F}_{n}$ be all finite sets with $n$ elements.\\
The \textbf{shatter coefficient} $s_{n}(\mathcal{A}) = \sup_{F \in \mathcal{F}_{n}} s(\mathcal{A},F) \leq 2^{n}$.\\
The \textbf{VC dimension} $d(\mathcal{A}) =$ the largest $n$ s.t. $s_{n}(\mathcal{A}) = 2^{n}$.\\
\textbf{Thm 5:} $\forall \epsilon>0$, $\mathbb{P}(\sup_{A \in \mathcal{A}} |P_{n}(A) - P(A)| > \epsilon) \leq 8 s_{n}(\mathcal{A})e^{-n\epsilon^{2}/32}$


\subsection*{Random Samples}
For $X_{1},\ldots,X_{n} \sim F$ a \textbf{statistic} is any $T = g(X_{1},\ldots,X_{n})$.\\
E.g. $\overline{X}_{n}$, $S_{n} = \sum_{i}(X_{i}-\overline{X}_{n})^{2} / (n-1)$, $\left(X_{(1)},\ldots,X_{(n)}\right)$\\
\textbf{Notes:} $\mathbb{E}(\overline{X}_{n}) = \mathbb{E}(X_{i})$, $\text{Var}(\overline{X}_{n}) = \text{Var}(X_{i})/n$, $\mathbb{E}(S_{n})^{2} = \text{Var}(X_{i})$, $X_{1,\ldots,n}$ $\sim$ $\text{Bern}(p)$ $\implies$ $\sum_{i} X_{i}$ $\sim$ $\text{Bin}(n,p)$, $X_{1,\ldots,n} \sim \text{Exp}(\beta)$ $\implies$ $\sum_{i}X_{i}$ $\sim$ $\Gamma(n,\beta)$, $X_{1,\ldots,n} \sim \mathcal{N}(0,1)$ $\implies$ $\sum_{i}X_{i}^{2} \sim \chi_{n}$. \\
\textbf{Thm. 1}: $X_{1},\ldots,X_{n} \sim \mathcal{N}(\mu,\sigma^{2})$ $\implies$  $\overline{X}_{n} \sim \mathcal{N}(\mu, \sigma^{2}/n)$.


\subsection*{Convergence}
$X,X_{1},X_{2},\ldots$ random variables.\\
(1) $X_{n}$ converges \textbf{almost surely} $X_{n} \xrightarrow{a.s.} X$ if $\forall \epsilon>0$
\\ 
\\
(2) $X_{n}$ converges \textbf{in probability} $X_{n} \xrightarrow{p} X$ if $\forall \epsilon>0$
 \\ \\
(3) $X_{n}$ converges \textbf{in quadratic mean} $X_{n} \xrightarrow{qm} X$ if
 \\ \\
(4) $X_{n}$ converges \textbf{in distribution} $X_{n} \rightsquigarrow X$ if
 \\ \\
$\forall t$ on which $F_{X}$ is continuous.\\

\textbf{Thm 7:} 

\textbf{Thm 10a:} $X$,$X_{n}$,$Y$,$Y_{n}$ random variables. Then
\\ 
\\
\\
\textbf{Thm 10b (Slutzky's Thm):} $X$,$X_{n}$,$Y_{n}$ random variables. Then
\\
\\
\textbf{Thm 12 (Law of Large Numbers):} $X_{1},\ldots,X_{n}$ iid, $\mathbb{E}(X_{i})=\mu$ $\implies$ $\overline{X}_{n} \xrightarrow{\text{qm}} \mu$.

\textbf{Thm 14 (CLT):} $X_{1},\ldots,X_{n}$ iid, $\mathbb{E}(X_{i})=\mu$ $\text{Var}(X_{i}) = \sigma^{2}$\\
$\implies$ $\sqrt{n}(\overline{X}_{n}-\mu)/\sigma \rightsquigarrow \mathcal{N}(0,1)$\\
$\implies$ $\overline{X}_{n} \rightsquigarrow \mathcal{N}(\mu,\sigma^{2}/n)$\\
$\implies$ $\sqrt{n}(\overline{X}_{n}-\mu)/S_{n}\rightsquigarrow \mathcal{N}(0,1)$

\textbf{Thm 18 (delta method):} If $\sqrt{n}(Y_{n}-\mu)/\sigma \rightsquigarrow \mathcal{N}(0,1)$, $g'(\mu) \neq 0$
$\implies$ $\sqrt{n}(g(Y_{n})-g(\mu))/|g'(\mu)|\sigma \rightsquigarrow \mathcal{N}(0,1)$\\
ie $Y_{n} \approx \mathcal{N}(\mu,\sigma^{2}/n)$ $\implies$ $g(Y_{n}) \approx \mathcal{N}(g(\mu),g'(\mu)^{2}\sigma^{2}/n)$

\textbf{Thm 18b (2nd order delta method):}


\subsection*{Sufficiency}
If $X_{1},\ldots,X_{n} \sim p(x;\theta)$, $T$ \textbf{sufficient} for $\theta$ if $p(x^{n}|t;\theta)$ $=$ $p(x^{n}|t)$.
\textbf{Thm 9 (factorization):} for $X^{n} \sim p(x;\theta)$, $T(X^{n})$ sufficient for $\theta$ if the joint probability can be factorized as.
\\

$T$ is a \textbf{minimal sufficient statistic (MSS)} if $T$ is sufficient and $T = g(U)$ for all other sufficient stats $U$.\\
\textbf{Thm 15:} $T$ is a MSS if:
\\

\subsection*{Parametric Point Estimation}
\textbf{Method of Moments:} Define equations\\
\\
And solve for $\hat{\theta}$.\\
\textbf{Maximum Likelihood (MLE):} The MLE is
\\

Often suffices to solve for $\theta$ in $\frac{\partial l(\theta)}{\partial \theta} = 0$.
The MLE is \textbf{equivariant} $\implies$ if $\eta = g(\theta)$ then $\hat{\eta} = g(\hat{\theta})$. \\
\textbf{Bayes Estimation:} For prior $\pi(\theta)$, choose 
\\

\textbf{Mean Squared Error (MSE):} The MSE is
\begin{equation}
    \text{MSE} = \mathbb{E}(\hat{\theta} - \theta)^{2} = \int (\hat{\theta}-\theta)^{2} p(x^{n};\theta)dx^{n} = \text{bias}({\hat{\theta}})^{2} + Var(\hat{\theta})
\end{equation}
Defs: $\text{\textbf{bias}}(\hat{\theta})$ $=$ $\mathbb{E}(\hat{\theta}) - \theta$. We say $\hat{\theta}$ is \textbf{consistent} if $\hat{\theta} = \hat{\theta}_{n} \xrightarrow{p} \theta$. The \textbf{standard error} of $\hat{\theta}$, $\text{se}(\hat{\theta})$, is the standard deviation of $\hat{\theta}$.\\


\subsection*{Risks and Estimators}
$L(\theta,\hat{\theta})$ is the \textbf{loss} of an estimator $\hat{\theta} = \hat{\theta}(x^{n})$ for $x^{n} \sim p(x^{n};\theta)$.\\
The \textbf{risk} of this $\hat{\theta}$ is
\\

When $L(\theta,\hat{\theta}) = (\theta-\hat{\theta})^{2}$, the risk is the MSE.\\
The \textbf{max risk} of $\hat{\theta}$ over a set $\theta \in \Theta$ is
\\

The \textbf{minimax risk} is
\\

The \textbf{minimax estimator} is
\\

The \textbf{Bayes risk} of $\hat{\theta}$ given a prior $\pi(\theta)$ is
\\

The \textbf{posterior risk} of $\hat{\theta}$ given a prior $\pi(\theta)$ is
\\
\\

where $\pi(\theta|x^{n}) = \frac{\mathbb{P}(x^{n};\theta)\pi(\theta)}{m(x^{n})}$ is the posterior over $\theta$.\\
The \textbf{Bayes estimator} is
\\

which equals the posterior mean $\mathbb{E}(\theta|x^{n})$ when $L(\theta,\hat{\theta}) = (\theta-\hat{\theta})^{2}$, the posterior median when $L(\theta,\hat{\theta}) = |\theta-\hat{\theta}|$, and the posterior mode when $L(\theta,\hat{\theta}) = \mathbb{I}[\theta \neq \hat{\theta}]$.\\
\textbf{Thm 10:} If $\hat{\theta}$ is a Bayes estimator for some prior $\pi$ and $R(\theta,\hat{\theta})$ is constant, then $\hat{\theta}$ is a minimax estimator.\\
\textbf{Note:} The MLE is approximately minimax (as n increases, if dimension of the parameter is fixed).


\subsection*{Distributions}
Discrete distributions:
(a) \hspace{2mm} \text{Bernoulli} \\
(b) \hspace{2mm} \text{Binomial} \\
(c) \hspace{2mm} \text{Poisson} 

Continuous distributions:
$(b) \hspace{2mm} \text{Normal}$

\subsection*{Expected Values}
The \textbf{mean} or \textbf{expected value} of $g(X)$ is

Related properties and definitions:\\
(g) \hspace{2mm} Cov(X,Y) = \\
(h) \hspace{2mm} Cov(X,Y) =  \\
(i) \hspace{2mm} $\rho(X,Y) = $\\
The \textbf{conditional expectation} of Y given X is the random variable $g(X) = \mathbb{E}(Y|X)$, where
\\

The \emph{Law of Total/Iterated Expectation} is\\
The \emph{Law of Total Variance} is\\
The \emph{Law of Total Covariance} is
\subsection*{Aymptotic (Large Sample) Theory}
A random sequence $A_{n}$ is:\\
1.\\
2.\\
3.\\
4.\\
If $Y_{n} \rightsquigarrow Y \implies Y_{n}=O_{p}(1)$\\
If $\sqrt{n}(Y_{n}-c)\rightsquigarrow Y$ $\implies$ $Y_{n}=O_{p}(1/\sqrt{n})$

\subsubsection*{Distances Between Distributions}
For distributions $P$ and $Q$ with pdfs $p$ and $q$:\\
 $\hspace{2mm} K(P,Q) = \int p \text{log}(p/q) \hspace{3mm} \text{\textbf{Kullback-Leibler} divergence}$ \\
A model is \textbf{identifiable} if: $\theta_{1} \neq \theta_{2}$ $\implies$ $K(\theta_{1},\theta_{2})>0$.

\subsubsection*{Consistency}
$\hat{\theta}_{n} = T(X^{n})$ is \textbf{consistent} for $\theta$ if $\hat{\theta}_{n} \xrightarrow{p} \theta$
    (ie if $\hat{\theta}_{n} - \theta = o_{p}(1)$).\\
To show consistency, can show: $\text{Bias}^{2}(\hat{\theta}_{n}) + \text{Var}(\hat{\theta}_{n}) \rightarrow 0$.\\
The MLE is consistent under regularity conditions.\\
MLE not consistent when number of params (or support?) grows.

\subsubsection*{Score and Fisher Information}
The \textbf{score function} is $S(\theta) = \frac{\partial}{\partial\theta} l(\theta) = \frac{\partial}{\partial\theta} \sum_{i=1}^{n} \text{log} \hspace{1pt} p(x_{i}|\theta)$.\\
The \textbf{Fisher information} is defined as
\begin{equation}
    I_{n}(\theta) = \mathbb{E}_{\theta} \left[ S(\theta)^{2} \right] = \text{Var}_{\theta} \left[ S(\theta) \right] 
        = -\mathbb{E}_{\theta} \left[ \frac{\partial^{2}}{\partial\theta^{2}} l(\theta) \right]
        %= -\mathbb{E}_{\theta} \left[ \frac{\partial}{\partial\theta} S(\theta) \right]
\end{equation}
    \newlinetab and $I_{n}(\theta) = -n\mathbb{E} \left[ \frac{\partial^{2}}{\partial\theta^{2}} \text{log}\hspace{1pt} p(X_{1};\theta) \right] = nI_{1}(\theta)$.\\
The \textbf{observed information} $\hat{I}_{n}(\theta) = -\sum_{i}\frac{\partial^{2}}{\partial\theta^{2}} \text{log}\hspace{1pt}p(X_{i};\theta)$.\\
    \newlinetab Vector case: $S(\theta) = \left[ \frac{\partial l(\theta)}{\partial \theta_{i}} \right]_{i=1,\ldots,K}$ \hspace{1mm}
        $I_{ij} = -\mathbb{E}_{\theta}\left[\frac{\partial^{2} l(\theta)}{\partial\theta_{i}\partial\theta_{j}}\right]_{i,j=1,\ldots,K}$

\subsubsection*{Efficiency and Robustness}
For an estimator $\hat{\theta}_{n}(X^{n})$ of $\theta$, where $X^{n} \stackrel{\text{iid}}{\sim} p(x|\theta)$:\\
If $\sqrt{n}(\hat{\theta}_{n} - \theta) \rightsquigarrow \mathcal{N}(0,v^{2})$, then $v^2$ is the \textbf{asymptotic-Var($\hat{\theta}_{n}$)}.\\
    \newlinetab E.g. for $\hat{\theta}_{n} = \overline{X}_{n}$:
        \hspace{1pt} $v^{2} = \sigma^{2} = \text{Var}(X_{i}) = \lim_{n \rightarrow \infty} n\text{Var}(\overline{X}_{n})$.\\
    \newlinetab In general, asymptotic-Var($\hat{\theta}_{n}$) $v^{2}$ $\neq$ $\lim_{n \rightarrow \infty} n\text{Var}(\hat{\theta}_{n})$.\\
    \newlinetab We will use approx: $\text{Var}(\hat{\theta}_{n}) \approx v^{2}/n$.\\
For param $\tau(\theta)$, $v(\theta) = \frac{|\tau'(\theta)|^{2}}{I_{1}(\theta)}$ is the \textbf{Cramer-Rao lower bound}.\\
    \newlinetab for most estimators $v^{2} \geq v(\theta)$.\\
If $\sqrt{n}(\hat{\theta}_{n}-\tau(\theta)) \rightsquigarrow \mathcal{N}(0,v(\theta))$ (ie if $v^{2} = v(\theta)$) $\implies \hat{\theta}_{n}$ \textbf{efficient}.\\
    \newlinetab usually, $\sqrt{n}(\tau(\hat{\theta}_{\text{mle}}) - \tau(\theta)) \rightsquigarrow \mathcal{N}(0,v(\theta))$ $\implies$ MLE efficient.\\
The \textbf{standard error} of \textbf{efficient} $\hat{\theta}_{n}$ is $se = \sqrt{\text{Var}(\hat{\theta}_{n})} \approx \sqrt{\frac{1}{I_{n}(\theta)}}$.\\
The \textbf{estimated standard error} of \textbf{efficient} $\hat{\theta}_{n}$ is $\hat{se} \approx \sqrt{\frac{1}{I_{n}(\hat{\theta}_{n})}}$.\\
    \newlinetab For efficient $\hat{\theta}_{n}$, $\hat{\tau} = \tau(\hat{\theta}_{n})$, $se \approx \sqrt{\frac{|\tau'(\theta)|^{2}}{I_{n}(\theta)}}$,
        and $\hat{se} \approx \sqrt{\frac{|\tau'(\hat{\theta}_{n})|^{2}}{I_{n}(\hat{\theta}_{n})}}$.\\
In general, \textbf{asymptotic normality} is when:\\
    \newlinetab $\frac{\hat{\theta}_{n} - \mathbb{E}(\hat{\theta}_{n})}{\sqrt{\text{Var}(\hat{\theta}_{n})}} \rightsquigarrow \mathcal{N}(0,1)$
        $\implies$ $\hat{\theta}_{n} \rightsquigarrow \mathcal{N}(\mathbb{E}(\hat{\theta}_{n}), \text{Var}(\hat{\theta}_{n}))$. \\
If $\sqrt{n}(W_{n}-\tau(\theta)) \rightsquigarrow \mathcal{N}(0,\sigma^{2}_{W})$ and
    $\sqrt{n}(V_{n}-\tau(\theta)) \rightsquigarrow \mathcal{N}(0,\sigma^{2}_{V})$ \\
    $\text{ }\hspace{2mm}$$\implies$ \textbf{asymptotic relative efficiency} $\text{ARE}(V_{n},W_{n}) = \sigma^{2}_{W} / \sigma^{2}_{V}$.\\
Often there is a tradeoff between efficiency and robustness. (?)

\subsection*{Hypothesis Testing}
\textbf{Null hypothesis} $H_{0}: \theta \in \Theta_{0}$, \textbf{alternative} $H_{1}: \theta \in \Theta_{1}$.\\
\textbf{Type I error}: If $H_{0}$ true but we reject $H_{0}$.\\
To construct a test:
\begin{equation}
    \begin{split}
        &1. \text{ Choose a test statistic } W = W(X_{1},\ldots,X_{n})\\
        &2. \text{ Choose a rejection region } R\\
        &3. \text{ If } W\in R, \text{ reject } H_{0} \text{ otherwise retain } H_{0}
    \end{split}
\end{equation}
For rejection region $R$, the \textbf{power function} $\beta(\theta) = \mathbb{P}_{\theta}(X^{n} \in R)$.\\
Want \textbf{level-$\mathbf{\alpha}$} test ($\sup_{\theta \in \Theta_{0}} \beta(\theta) \leq \alpha$) that maximizes $\beta(\theta\in\Theta_{1})$.\\
A level-$\alpha$ test with power fn $\beta$ is \textbf{uniformly most powerful} if:
    \newlinetab$\beta(\theta) \geq \beta'(\theta)$ $\forall \theta \in \Theta_{1}$ $\forall \beta'\neq\beta$.
\subsubsection*{Neyman-Pearson Test}
For simple $H_{0}: \theta=\theta_{0}$ and $H_{1}: \theta=\theta_{1}$, reject $H_{0}$ if $\frac{L(\theta_{1})}{L(\theta_{0})} > k$.
    \newlinetab where $k$ chosen s.t. $\mathbb{P}(\frac{L(\theta_{1})}{L(\theta_{0})} > k) = \alpha$.
\subsubsection*{Wald Test}
For $H_{0}: \theta=\theta_{0}$ and $H_{1}: \theta\neq\theta_{0}$, reject $H_{0}$ if $\left| \frac{\hat{\theta}_{n} - \theta_{0}}{se} \right| > z_{\alpha/2}$.\\
    \newlinetab where $z_{\alpha/2}$ is the inverse standard-normal CDF of $1-\frac{\alpha}{2}$. \\%, ie $\Phi^{-1}(1-\alpha/2)$
    \newlinetab and $\hat{\theta}_{n}$ is an unbiased estimator for $\theta$.\\
    \newlinetab and $se = \sqrt{\text{Var}(\hat{\theta}_{n})}$. Can also use $\hat{se} =_{\text{eg.}} \sqrt{S_{n}^{2}/n}$.\\
    \newlinetab and if $\hat{\theta}_{n}$ efficient, can approx: $se \approx \sqrt{\frac{1}{I_{n}(\theta)}}$ or $\hat{se} \approx \sqrt{\frac{1}{I_{n}(\hat{\theta}_{n})}}$.

\subsubsection*{Likelihood Ratio Test}
For $H_{0}: \theta\in\Theta_{0}$ and $H_{1}: \theta\notin\Theta_{0}$, reject $H_{0}$ if $\lambda(x^{n}) = \frac{L(\hat{\theta}_{0})}{L(\hat{\theta})} \leq c$.
    \newlinetab where $L(\hat{\theta}_{0}) = \sup_{\theta\in\Theta_{0}}L(\theta)$ and $L(\hat{\theta}) = \sup_{\theta\in\Theta}L(\theta)$.\\
    \newlinetab and $c$ chosen s.t. $\mathbb{P}(\lambda(x^{n}) \leq c) = \alpha$.\\
    \newlinetab \textbf{Thm:} under $H_{0} : \theta=\theta_{0}$ $\implies$ $W_{n} = -2\text{log}\lambda(X^{n}) \rightsquigarrow \chi_{1}^{2}$\\
    \newlinetab\newlinetab $\implies$ reject $H_{0}$ if $W_{n}>\chi_{1,\alpha}^{2}$.\\ %\lim_{n\rightarrow\infty} \mathbb{P}_{\theta_{0}}(W_{n} > \chi_{1,\alpha}^{2}) = \alpha$.\\
        \newlinetab\newlinetab Also: for $\theta=(\theta_{1},\ldots,\theta_{k})$, if $H_{0}$ fixes some of the parameters\\
        \newlinetab\newlinetab $\implies$ $-2\text{log}\lambda(X^{n}) \rightsquigarrow \chi_{\nu}^{2}$, where $\nu = \text{dim}(\Theta) - \text{dim}(\Theta_{0})$.

\subsubsection*{P-Values}
The \textbf{p-value} $p(x^{n})$ is the smallest $\alpha$-level s.t. we reject $H_{0}$.\\
\textbf{Thm:} For a test of the form: reject $H_{0}$ when $W(x^{n})>c$,\\
    $\text{ }\hspace{1mm}$ $\implies$ $p(x^{n}) = \underset{\theta\in\Theta_{0}}{\sup} \mathbb{P}_{\theta}(W(X^{n}) \geq W(x^{n}))$
    $=$ $\underset{\theta\in\Theta_{0}}{\sup} [1 - F(W(x^{n})|\theta)]$.\\
\textbf{Thm:} Under $H_{0}:\theta=\theta_{0}$,\hspace{2mm}$p(x^{n}) \sim \text{Unif}(0,1)$.

\subsubsection*{Permutation Test}
$X^{n} \sim F$, $Y^{m} \sim G$, $H_{0}:F=G$, $H_{1}:F \neq G$\\
Let $Z=(X^{n},Y^{m})$ and $L=(1,\ldots,1,2,\ldots,2)$.\\
Let $W = g(L,Z) = |(\text{ave of 1 labeled pts}) - (\text{ave of 2 labeled pts})|$.
Let $p = \frac{1}{N!}\sum_{\pi} \mathbb{I}\left( g(L_{\pi},Z) > g(L,Z) \right)$ $\implies$ reject $H_{0}$ when $p<\alpha$.

\end{document}
